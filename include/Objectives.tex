\chapter{Aims of the project}

The production and commercialization of chemicals have raised concerns regarding their potential impact on public health and the environment. Endocrine disruptors have been identified as a significant concern due to their ability to interfere with the normal functioning of the endocrine system, which regulates hormones in the body. To assess their toxicity, traditional and in-vitro high-throughput assays have been applied to pure substances. However, these approaches can be time-consuming and difficult to implement in mixtures and complex samples for which the constituents are not determined, such as environmental samples.
 
Mass spectrometry can capture the fragmentation patterns of molecules, providing valuable analytical information that can potentially reveal the toxicity potency of these chemicals within samples. However, the use of \tMS{} data for predicting endocrine activity has not been explored to date. This study proposes two main hypotheses: (1) chemicals that are closely connected within a spectral similarity network exhibit comparable endocrine disruptive activity, and (2) the similarity of \tMS{} spectra can be utilized to predict the endocrine disruptive activity of unknown chemicals.

The primary objective of this study was to predict the endocrine disruptive potency of chemicals detected with LC-ESI-HRMS, based on the similarity of their \tMS{} spectra. To achieve this goal, the following specific objectives have been outlined:

\begin{enumerate}
\item[\hspace{1cm}I.]  To obtain, combine and process \tMS{} spectra for the chemicals included in the Tox21 Challenge.
\item[\hspace{1cm}II.] To build spectral similarity networks and analyze their ability to connect mass spectra with similar endocrine activity.
\item[\hspace{1cm}III.] To predict endocrine activity from \tMS{} spectra using \textit{k}-nearest neighbors and spectral similarity networks.
\end{enumerate}

By achieving these objectives, this study explores the prediction of toxicity from \tMS{} spectra based on mass spectral networks, thereby contributing to the understanding of the application of mass spectra in toxicity prediction.



