\chapter{Conclusions and future perspectives}

In this study, the application of a mass spectra similarity networking was studied for predicting toxicity endpoints based on tandem mass spectra. First a \textit{k}-NN classification was explored to determine the ability of predicting active labels in the whole dataset. The classification results showed low recall and precision, being the highest for NR.AR at 47.1 \% and 44.4 \%, respectively. The implementation of mass spectra networks showed a better predictive ability for the same endpoint, with recall and precision of 81.8 \% and 75.0 \%, respectively. The definition of the applicability domain using the networks narrows the chemical space and improves the recall and precision. The higher tendency of clustering within the NR.AR endpoint can be due to the prevalence of the steroid core structure in the dataset. The results showed promising outcomes in terms of distinguishing between active and inactive compounds for NR.AR, NR.AR.LBD, NR.ER and NR.ER.LBD, suggesting that the spectral similarity networks can capture meaningful relationships among chemicals based on their mass spectra that can be used to predict activity of toxicity endpoints. 

To improve future predictions, several aspects can be considered. First, expanding the dataset to include a larger number of diverse compounds could potentially enhance the model's performance by capturing a broader range of chemical activity and classes. Additionally, incorporating feature selection techniques, such as machine learning algorithms or ensemble methods, could help identifying the most informative fragments related to specific endpoints. Moreover, exploring different similarity metrics and optimizing the spectral networking parameters may further enhance the performance of the predictions.  Lastly, adopting systems biology approaches can provide valuable insights into the mechanisms underlying toxicity that is not related to chemical structure alone. By integrating data from multiple domains, a more comprehensive understanding of the molecular basis of toxicity can enhance future predictions.
