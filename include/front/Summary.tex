\thispagestyle{plain}			% Supress header 
\setlength{\parskip}{0pt plus 1.0pt}
\phantomsection
\section*{Popular summary}
\addcontentsline{toc}{chapter}{Popular summary}

The brain has a great ability to classify information. It organizes and categorizes a vast amount of sensory data based on shared characteristics and similarities. When examining a new object, the brain quickly identifies common features like shape, color, smell, texture, and size and then associates these features with known categories. 

At the same time, there is a vast universe of chemicals that is expanding. Some of them have adverse effects to humans and have been categorized as toxic. Among these chemicals, the endocrine disruptors are of particular concern because they interfere with the normal functioning of hormones in the body, potentially leading to health issues. 

The identification of these chemicals in samples can be time-consuming and challenging. One of the most sensitive analytical techniques used for this purpose is tandem mass spectrometry. Tandem mass spectrometry, like solving a complex puzzle from small pieces, weights molecules and uses their fragments to infer its identity. Even though the fragments are detected, it is not always possible to fully identify all the chemicals in a sample.

To predict the endocrine disruptive activity of chemicals in a sample, this study mimics the brain classification ability and applies it to chemical toxicity. A digital brain builds a network of known chemical toxicity based on a vast amount of tandem mass spectrometry and toxicity data. Then, this network is used to interpret the fragments of unknowns and to infer their toxicity. 

In this way, this study advances the analytical capabilities in chemical toxicity by combining mass spectrometry and a network-based approach. Ultimately, it can aid in the assessment of chemical risk and provide valuable insights for environmental monitoring and public health protection.

