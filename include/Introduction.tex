\chapter{Introduction}

The increasing production of chemicals has brought numerous benefits to society, from life-saving medicines to essential materials and components. However, the production and commercialization of new chemicals have raised concerns about their potential harmful effects on human health and the environment \cite{Landrigan2018}. Therefore, regulations have been put in place to ensure their safe use. For instance, the European Union's Registration, Evaluation, Authorization, and Restriction of Chemicals (REACH) aims to ensure the safe production, commercialization, and use of chemicals by assessing their risks to human health and the environment. The database have listed thousands of chemicals with their properties and toxicity information \cite{ECHA}. Nevertheless, the understanding of the effects of these chemicals is limited, and research efforts are ongoing to uncover their distribution and potential risks \cite{Escher2020}\cite{Ebele2017}\cite{Sill2020}.

Efforts to characterize the toxicity of chemicals have traditionally relied on animal testing and epidemiological studies, which are time-consuming, expensive, and often raise ethical concerns. Emerging technologies, such as high-throughput in-vitro testing, high-content data analysis (omics, image analysis), and bioinformatics, are transforming the traditional approach towards a mechanistic understanding of toxicity \cite{Krewski2010}. However, these approaches mostly rely on the identified compounds and toxicity in complex mixtures are more difficult to assess.

\ac{MS} is an analytical technique that can identify and quantify chemicals in complex matrices. Regarding toxic chemicals, \ac{MS} can provide a more comprehensive understanding of their distribution and fate in the environment \cite{Richardson2011}. However, identifying and quantifying all toxic chemicals in a sample is indeed a challenging task \cite{rager_linking_2016} due to the complexity of sample matrices, the wide range of chemical substances involved, and the need for suitable chemical standards.


At the same time, several large mass spectra databases have been expanding their repositories, primarily through collaborative efforts, over the past decade (e.g., MassBank Europe\cite{Horai2010}, MoNA\cite{mona_website}, HMDB\cite{Wishart2017}, GNPS\cite{Wang2016}). These databases offer opportunities for data mining and the exploration of prediction models. Very recently, it has been proposed that toxicity relevant information can also be gained from the mass spectra collected in chemicals analysis of complex samples \cite{Zushi2022}\cite{peets_MS2tox_2022}. 

The lack of comprehensive toxicological data for all chemicals compounds and the difficulty to identify all chemicals in environmental samples challenges the assessing of potential risks \cite{Grandjean2014}\cite{Vandenberg2016}. To address these issues, this study seeks to directly predict the endocrine activity of detected chemicals using LC-ESI-HRMS data. By utilizing this approach, it is hoped that in the future, prioritizing toxic samples for further analysis will become more efficient, while extensive screening of contaminated environmental areas or commercialized products will become more feasible. Ultimately, such efforts will help to identify and mitigate potential risks to human health and the environment.
